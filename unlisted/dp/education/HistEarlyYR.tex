\documentclass[letterpaper]{article}

%\usepackage[pdftex]{hyperref}
%\usepackage{hyperref}
\usepackage{geometry}
%\usepackage{fontspec}
\usepackage{xcolor}
%\usepackage{textcomp}
\usepackage{fancyhdr}
\usepackage{graphicx}
\usepackage{parskip}
%\usepackage{textcomp}
\usepackage{amsmath,amssymb}

\newcommand{\smallspace}{\hspace{0.5ex}}
\newcommand{\tridot}{\smallspace{}\raisebox{0.3ex}{\textbf{{\small$\pmb{\boldsymbol{\therefore}}$}}}\smallspace{}}

\author{V\tridot{}I\tridot{} Companion Mark D. Myer, PGM}
\title{Early History of the York Rite in Texas}
\date{January 4, 2001}

\begin{document}
	
\maketitle

Although membership in the Masonic fraternity has been declining since the early 1960’s, the 1999
Proceedings of the Grand Lodge of Texas reported over 134 thousand Masons (134,477) in good
standing in Texas. Yet, at the same time, there were only about 17 thousand of those same Masons
who were members of the Texas York Rite Bodies, or about 12 ½\% of the Masonic population. Of
course, we could expound for hours about the many reasons why there aren’t more Texas Masons
in the York Rite, but I think the discussion would eventually lead us to one glaring conclusion.
We, as Masons, and particularly York Rite Masons, do not know very much about our York Rite
heritage in this state.

Masonic scholars have done a good job of piecing together the obscure historical threads of the
oldest fraternity in the world. When we talk to a non-Mason about our fraternity, we can’t help
but feel a spark of pride when we tell him about some of the more famous members of our Masonic
family, such as George Washington, Benjamin Franklin, the Marquis de Lafayette, Andrew Jackson,
Theodore and Franklin D. Roosevelt, Harry Truman and many others. Most of us are very familiar
with the key role that Masonry played in giving birth to the Republic of Texas. We feel honored to
be associated with such Masons as Stephen F. Austin, Sam Houston, Mirabeau B. Lamar, James
Fannin, William B. Travis, James Bowie and David Crockett, some of whom paid the ultimate price
to uphold and preserve the principles that are the foundation of our Masonic beliefs.

Despite all of this, however, we suddenly find ourselves at a loss for words when we attempt to
explain to a brother Mason about the benefits of York Rite membership and the beautiful lessons
illustrated by the Degrees. We have no historical background in our memory that we can call
upon to explain our existence. In attempting to do some research on the subject, one can see why
this condition of historical ignorance exists. To date, there has been very little written about the
beginnings of the York Rite in America, and almost nothing at all about Texas York Rite Masonry.
However, an examination of the early proceedings of the Grand Chapter and Grand Council of
Texas reveals some interesting and surprising facts.

To place events in their proper perspective, it is important to look at several key historical events,
both Masonic and political, in Texas history. Early in 1832 the arrest of William B. Travis and two
other Masons by Mexican military authorities near Anahuac triggered a conflict, which is considered
to be the beginning of the Texas Revolution. As political events surrounding the Texas struggle for
independence began to unfold, the seeds of Texas Masonry were beginning to take root as well. In
March of 1835 the now famous meeting under the oak tree at Brazoria was held, in which several
prominent Masons, including Anson Jones, agreed to request a dispensation from the Grand Lodge
of Louisiana to form a Lodge in Texas. The Dispensation was granted by John Henry Holland, then Grand Master of the Grand Lodge of Louisiana, and on December 27, 1835, Holland Lodge No. 36 was instituted and opened in the second story of the old courthouse building in Brazoria.
A month later, on January 29, 1836, Holland Lodge No. 36 was granted a Charter by the Grand
Lodge of Louisiana with Anson Jones designated as the first Worshipful Master. John M. Allen of
Louisiana Lodge No. 32 delivered the Charter personally to Anson Jones just prior to the Battle
of San Jacinto. On December 9, 1835, three weeks prior to the first meeting of Holland Lodge in
Brazoria, a Charter was granted by the General Grand Chapter of the United States to San Felipe
de Austin Royal Arch Chapter No. 1 to form and meet at San Felipe de Austin. However, due to
unforeseen events, the Chapter was never opened there. Because of events surrounding the Texas
revolution, it was almost two years before Holland Lodge No. 36 was able to hold another meeting.

On March 2, 1836, at a Convention assembled at Washington-on-the-Brazos, a Declaration of
Independence was drawn up and signed, declaring Texas’ independence from Mexico. Two days
later, on March 4, 1836, a Constitution of the Republic of Texas was drafted and an ad interim
government created and installed with David G. Burnet as the first President. Sam Houston was
also elected commander-in-chief of all Texas land forces. While this Convention was in session the
Alamo was under siege by the Mexican army under Santa Anna. Two days after the ad interim
government was formed, the Alamo fell, on March 6, 1836, and its defenders were massacred. On
April 21, 1836, one and a half months after the historic stand by the Texas defenders at the Alamo,
Sam Houston, commanding an army of 783 Texans and volunteers, won the decisive Battle at San
Jacinto against Santa Anna and a Mexican army numbering well over 1200 men. Santa Anna
himself was captured the next day and held as prisoner until the end of the Texas conflict a few
months later. On September 1, 1836 a general election was held, Sam Houston was elected President
and the Constitution of the Republic of Texas was ratified. It is interesting to note that in 1835
there were only about 300 Masons in Texas. Although Masons comprised less than two percent of
the total population, they came to occupy over 41\% of the seats in the Senate, 47\% in the House of
Representatives, 88\% of the chief executive offices, and 60\% of the principal judicial offices in the
first constitutional government of the Republic of Texas. Texas remained a sovereign Republic for
only nine years, until it was annexed to the United States of America on December 29, 1845.

Due to the many tasks at hand for the new Republic of Texas, it was another year before Masonry
saw any prominent activity. In late August or early September of 1837 McFarland Lodge No.
41 at San Augustine was set to work under dispensation from the Grand Lodge of Louisiana.
Their Charter, dated September 22, 1837, was received some time later. At the same time, Milam
Lodge No. 40 at Nacogdoches was issued a Charter with the same date from the Grand Lodge of
Louisiana. A little over a month later, on November 8, 1837, Holland Lodge No. 36, with Anson
Jones as Worshipful Master, reconvened in Houston for the first time since its original meeting at
Brazoria. On December 20, 1837, representatives of Holland Lodge No. 36, Milam Lodge No. 40
and McFarland Lodge No. 41 met in convention in Houston to establish the Grand Lodge of Texas.
The convention elected Anson Jones the first Grand Master, a committee was appointed to draw
up a constitution, and the first meeting of the Grand Lodge of the Republic of Texas was called to
meet at Houston on April 16, 1838, at which time the jurisdiction of the Grand Lodge of Louisiana
was ended.

Prior to the formation of the Grand Chapter of Texas, Chapters of Royal Arch Masons were formed
and operated as Masonic Bodies appendant to a Blue Lodge and under the sanction and jurisdiction
of the Grand Lodge of the Republic of Texas. On June 2, 1840, San Felipe de Austin Chapter,
having never met at San Felipe de Austin, met and opened a Chapter of Royal Arch Masons at Galveston, Texas. The General Grand Chapter of the United States, from whom they had obtained
their Charter, did not grant approval for the removal to Galveston until September 12, 1844, some
four years later. Meanwhile, in 1841, Cyrus Chapter at Matagorda, Rising Star Chapter at San
Augustine and Lone Star Chapter at Austin petitioned for dispensations from the Grand Lodge
of the Republic of Texas, and on December 10, 1841 those dispensations were granted. Four days
later, on December 14, 1841, representatives of those Chapters, along with those of Washington
Chapter No. 2 in Houston, met in Convention in Austin to form the Grand Chapter of Royal
Arch Masons of the Republic of Texas. One week later, on December 21, 1841, a Constitution
was ratified and adopted. However, San Felipe de Austin Chapter refused to sign the Constitution
and withdrew from the Convention. Upon receipt of a letter informing them of the formation of
the Grand Chapter and requesting relinquishment of jurisdiction over local Chapters and Royal
Arch Masons, the Grand Lodge of the Republic of Texas adopted a resolution surrendering said
jurisdiction and granting recognition to the Grand Royal Arch Chapter of the Republic of Texas
as the appropriate head and governing body of Texas Capitular Masonry. However, the General
Grand Chapter of the United States refused to recognize the legal authority of the Grand Chapter
of Texas. The differences between the two governing bodies were never satisfactorily resolved and in
1847, six years after the formation of the Grand Royal Arch Chapter of Texas, the General Grand
Chapter passed a resolution forbidding Masonic intercourse between any Chapters and Royal Arch
Masons under its jurisdiction, and those of the Grand Chapter of Texas.

Two years later, in 1849, in an attempt to preserve peace and harmony among the craft, the Grand
Royal Arch Chapter of Texas adopted a resolution to dissolve itself. During all of this confusion,
eight of the nine Chapters of Texas petitioned for dispensations from the General Grand Chapter to
open their Chapters under authority and jurisdiction of the General Grand Chapter of the United
States. These dispensations were granted between 1848 to 1850 and six of the eight Chapters
subsequently received Charters. One year after dissolving itself, representatives from four Texas
Chapters met in Galveston, on December 30, 1850, to again form the Grand Chapter of Texas, but
this time under authority of the General Grand Chapter of the United States. All but San Felipe
de Austin agreed to surrender their authority from the General Grand Chapter and place it under
the Grand Chapter of Texas. Charters from the reinstituted Grand Chapter of Texas were granted
on June 25, 1851. San Felipe de Austin Chapter did not receive a Charter from the Grand Chapter
of Texas until June 22, 1860. On June 17, 1861, no doubt precipitated by the cessation of the state
of Texas from the Union during the Civil War, the Grand Chapter of Texas adopted a resolution
permanently dissolving any connection with the General Grand Chapter of the United States. This
dissolution remains to this day.

The Grand Council of Royal and Select Masters of Texas was born from a Convention held on
June 24, 1856 in Huntsville, Texas, just 18 years after the first meeting of the Grand Lodge of
the Republic of Texas. Representatives from Houston Council No. 10 (chartered from New York),
Austin Council No. 12 (chartered from Alabama), Galveston Council (granted a dispensation
from Alabama), and Coleman Council (located in Marshall, Texas; chartered from Kentucky) were
present. It is interesting to note that Companion Andrew Neill, who was elected as the first Grand
Recorder, was also a Charter member of the Grand Lodge of the Republic of Texas. In the period
from 1857 to 1860 the Grand Council of Texas enjoyed a healthy growth, having issued 14 Charters
to Councils in Texas and 1 Charter to Marysville Council in California. It seems that events
leading up to the Civil War began to take their toll on Grand Council activities from 1861 to
1864, as reflected in Grand Council proceedings. Finally, at the Grand Assembly of the Grand Council of Royal and Select Masters of Texas held in Houston on June 16, 1864, a resolution was
adopted by the representatives present to dissolve the Grand Council, and to transfer jurisdiction
of the Council Degrees, subordinate Councils and their members, and all funds and properties of
the Grand Council of Texas to the Grand Royal Arch Chapter of Texas. The existing Councils
were then made appendant bodies of Royal Arch Chapters in Texas. Later, in the years following
1877, the action of the Grand Council of Texas to dissolve itself was followed in kind in eight
other states, namely Mississippi, Arkansas, Illinois, Kentucky, Wisconsin, Iowa, South Carolina,
and North Carolina. However, after a few years trial, it was in every instance abandoned, and
jurisdiction released by the Grand Chapters and resumed by the Grand Councils.

During the 43 years following the demise of the Grand Council of Texas, Cryptic Masonry experienced
a steady growth under the jurisdiction of the Grand Chapter, eventually establishing
appendant Councils for each of the more than 203 Chapters of Royal Arch Masons in Texas. By
1907, of the nearly 12,000 Royal Arch Masons then in Texas, almost all had become Royal and
Select Masters. Despite this phenomenal growth, the Grand Chapter of Royal Arch Masons, in its
Annual Convocation at Waco, Texas, on December 3, 1907, adopted a resolution requesting that
the Grand Council of Royal and Select Masters of Texas should reconvene and again take jurisdiction
over the Degrees of Royal and Select Master. On that same day a Convention of Royal and
Select Masters, attended by representatives of 120 Texas Councils, was held at Waco, Texas, when
it was decided to comply with the request of the Grand Chapter. A resolution was subsequently
adopted by the delegates in attendance establishing rehabilitation of the Grand Council of Royal
and Select Masters of Texas. As of this writing it has continued to operate in peace and prosperity
for 94 years since rehabilitation. I pray that it, and all of Texas York Rite Masonry, will continue
to prosper, but only through the hard work and persistent efforts of dedicated Masons such as you
here today can it do so. May God bless you in your efforts. Thank you.

\subsection*{References}

Carter, James D., \textit{Masonry in Texas}, 1955

\textit{Proceedings of the Grand Council, Royal and Select Masters of Texas}, 1856-1864, 1907-1909

\textit{Transactions of the Grand Royal Arch Chapter of Texas}, Volumes I-III

\textit{Proceedings of the Grand Lodge of Texas}, 1999

\end{document}