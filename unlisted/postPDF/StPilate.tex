\documentclass[letterpaper,11pt]{article}

\usepackage{geometry}
\usepackage{xcolor}
\usepackage{fancyhdr}
\usepackage{graphicx}
\usepackage{parskip}
\usepackage{amsmath,amssymb}
\usepackage{textcomp}

\usepackage[backend=bibtex, style=geschichtsfrkl]{biblatex}
\addbibresource{StPilate}

\title{Giving Thanks for Pilate’s Accusation:\\\Large{A Perspective from the Maltese Priory}}
\author{SK Gabriel A. Jagush, Senior Warden\\Worth Commandery \textnumero{} 19, KT}
\date{\emph{Originally presented at the October 3, 2019\\Stated Conclave of Worth Commandery \textnumero{} 19, KT.}}

\begin{document}
	\maketitle
	
	According to the Gospels, when Pontius Pilate sentenced Jesus of Nazareth to execution by crucifixion, he ordered that a sign be placed over Jesus’ head while on the cross. The exact phrasing of the sign is unclear, but the most famous rendition is from John 19:19-20, which reads as follows:
	
	\begin{quote}
		Pilate also had an inscription written and put on the cross. It read, ``Jesus of Nazareth, the King of the Jews.''
	\end{quote}
	
	This title, which was a criminal accusation from Pilate, has survived to present day, and evolved into a title of unparalleled honor among Christians for Jesus of Nazareth, now Jesus the Christ. It is represented by the Latin initialism \emph{INRI}.
	
	\subsection*{History of the Term}
		
	There are two primary terms related to \emph{INRI}. The first is the title ``King of the Jews,'' which was used exclusively by Gentiles such as the Magi, Pilate, the Romans, and so forth. The Jewish population of Judea instead used the term ``King of Israel.'' Each group objected to the other group’s epithet towards Jesus for different reasons, while Pilate himself objected to the use of the term ``King,'' due to the implication of revolution against his governorship of Judea. The author of the Gospel of Mark himself makes a careful and conscious distinction between the two terms, and who uses them.
	
	The first reference to ``King of the Jews'' that we see is in Matthew 2:1-2, when the Magi talk to Herod, asking ``where is the child who has been born King of the Jews? For we observed his star at its rising, and have come to pay him homage.'' This sets a terrible series of events into movement. Herod tries to interrogate the Magi, but fails. His failure to discover the identity of the perceived pretender to the throne leads to a genocidal edict to kill all Bethlehemite males under the age of three.
	
	The first reference to either term in the Passion Narratives occurs during Jesus’ interrogation by Pilate. In each of the Gospels, Pilate asks Jesus, ``Are you the King of the Jews?'' (Matthew 27:11, Mark 15:2, Luke 23:3, John 18:33). In the three Synoptic Gospels (Matthew, Mark, \& Luke), Jesus replies, ``You say so.'' However, in John 18:33-37, the exchange is substantially different:
	
	\begin{quote}
		Then Pilate entered the headquarters again, summoned Jesus, and asked him, ``Are you the King of the Jews?'' Jesus answered, ``Do you ask this on your own, or did others tell you about me?'' Pilate replied, ``I am not a Jew, am I? Your own nation and the chief priests have handed you over to me. What have you done?'' Jesus answered, ``My kingdom is not from this world. If my kingdom were from this world, my followers would be fighting to keep me from being handed over to the Jews. But as it is, my kingdom is not from here.'' Pilate asked him, ``So you are a king?'' Jesus answered, ``You say that I am a king. For this I was born, and for this I came into the world, to testify to the truth. Everyone who belongs to the truth listens to my voice."
	\end{quote}
	
	After interrogating Jesus, Pilate has him scourged and humiliated. The soldiers clothe him in purple robes and a crown of thorns, mocking his status as a ``king'' (Matthew 27:29-30, Mark 15:17-19, John 19:2-3). The primary criminal charge leveled against Jesus is claiming to be a king (John 19:12). Once Jesus is crucified, some version of ``The King of the Jews'' is placed over his head (Mark 15:26, Luke 23:38, Matthew 27:37, John 19:19-20). According to some translations of Luke 23:28, such as the ones found in the 1599 Geneva Bible and the King James Version, it was specifically written in Latin, Greek, and Hebrew. The last use of ``King of the Jews'' occurs in Luke 23:36-37 and Matthew 27:42 when the Roman soldiers mock Jesus as he is dying on the cross.
		
	\subsection*{Use by the Church}
		
	The Early Church often referred to Jesus as the ``King of the Judeans.'' This was a huge risk for members to take, as this was tantamount to treason, and by calling themselves ``followers of Jesus,'' they were essentially associating themselves with a revolutionary agent. This was more strongly emphasised by Christ’s name, which we have Romanized as ``Jesus,'' but was originally \emph{Yeshua} or ``Joshua,'' and meant ``liberator.'' \cite{wren}
	
	As the Early Church evolved into the Western Church and the Eastern Church, so too did the initialism used on representations of the cross. The Western Church uses \emph{INRI,} which stands for the Latin phrase \emph{IESUS NAZARENUS REX IUDAEORUM.} The Eastern Church instead uses \emph{INBI,} which is the initialism of the Greek phrase, \emph{IESUS HO NAZORAEOS HO BASILEUS TON IUDAEON.} \cite{doornbos} The Greek word \emph{basileus} means ``monarch,'' usually in reference to a king or an emperor. \cite{doornbos} The Eastern Church also frequently uses the variant \emph{INBK,} for \emph{IESUS HO NAZORAEOS HO BASILEUS TU KOSMU,} which translates to English as ``Emperor of the Universe'' instead of ``King of the Jews.'' According to Catholic tradition, Saint Helena (who is revered by both the Western and Eastern Churches) brought the tablet with the Latin, Greek, and Hebrew inscriptions to Rome. \cite{weiss}
			
	\subsection*{Esoteric Latin Interpretations}
				
	There are a number of Latin sayings or mottos that have been generated from INRI. \cite{zeldis} Among them include:
	\begin{itemize}
		\item \emph{In Necis Renascor Integer} --- In Death, I Am Reborn Intact and Pure. 
		\item \emph{Iustum Necare Reges Impios} --- It is Just to Kill Impious Kings
		\item \emph{Igne Nitrum Roris Invenitur} --- By Fire, the Nitre of the Dew is Discovered
		\item \emph{Intra Nobis Regnum Iehova} --- The Kingdom of God is Within Us 
	\end{itemize}
	
	Of note are two particular versions which have been carried into many esoteric traditions, including some degrees in Freemasonry \cite{zeldis}:
	\begin{itemize}
		\item \emph{Igne Natura Renovatur Integra} --- By Fire, Nature Renews
		\item \emph{Insignia Naturae Ratio Illustrat} ---  Reason Illuminates Nature’s Symbols
	\end{itemize}
				
	\subsection*{Esoteric Hebrew Interpretations}
					
	Esoteric traditions often tie the letters of INRI to the Hebrew words \emph{yam,} \emph{nur,} \emph{ruach,} and \emph{yebeshas.} \cite{mackey} \emph{Yam} translates to ``vast body of water,'' and represents the element of water. \emph{Nur} translates to ``fire.'' \emph{Ruach} translates to ``breath'' or ``wind'' and represents air. According to Albert Mackey, \emph{yebeshas} was translated by Jean Baptiste Marie Ragon to mean ``earth'' \cite{mackey}, although there is little evidence that this is even a real Hebrew word. Regardless, if this is the correct interpretation, then the word \emph{INRI} is a representation of Jesus Christ as the Creator of all. 
	
	One of the more interesting interpretations of \emph{INRI} is drawn using letter-based correspondences from an 
	esoteric Hebrew text called the \emph{Book of Formation}, written some time between the 2nd Century BC and the 2nd Century AD. The \emph{Book of Formation} describes correspondences between Hebrew letters, elements, numbers, planets, and Zodiac signs. From \emph{Formation}, we can draw this cycle, as described in \emph{Modern Magick} by Donald Kraig \cite{kraig}:
	
	\begin{enumerate}
		\item ``I'' is tied to \emph{Yod}, which corresponds to Virgo. It represents untouched nature and birth.
		\item ``N'' is tied to \emph{Nun}, which corresponds to Scorpio. It represents death.
		\item ``R'' is tied to \emph{Resh}, which corresponds to the Sun. It represents light and resurrection.
		\item The final ``I'' once again represents untouched nature and birth.
	\end{enumerate}	

	This correspondence gives us the basic function of man’s journey to Christ. We are born in our natural state. We choose to die in Christ and be resurrected in Christ. We are reborn as new beings. This applies to both our emotional and spiritual journey in Christ while on Earth as well as our journey to and past Judgment Day. This is the \emph{process} of \emph{INRI}.
			
	\subsection*{Exactly What it Says on the Tin}
				
	The most important of the inscriptions above Christ’s head, however, may have been the one in Hebrew. It read \emph{Yshu Hnotsri Wmlk Hyhudim,} which, when initialized, gives us the letters ``Yod-Heh-Vav-Heh,'' the ineffable Hebrew name of God. Pilate, probably unknowingly and unintentionally, declared exactly who Jesus was to the world at large. When challenged by the Jewish leaders to change the inscription, he gave us the famous response, \emph{``Quod scripsi, scripsi,''} or, ``what I have written, I have written.''

	\vspace*{\fill}

	\pagebreak
	
	\printbibliography
	
\end{document}