\documentclass[letterpaper,11pt]{article}

\usepackage{geometry}
\usepackage{xcolor}
\usepackage{fancyhdr}
\usepackage{graphicx}
\usepackage{parskip}
\usepackage{amsmath,amssymb}
\usepackage{textcomp}
\usepackage{wrapfig}

\title{Our Cross to Bear}
\author{SK Brent C. Casiglio, Standard Bearer\\Worth Commandery \textnumero{} 19, KT}
\date{\emph{Originally presented at the November 7, 2019\\Stated Conclave of Worth Commandery \textnumero{} 19, KT.}}

\begin{document}
	
	\maketitle
	
	In our September stated conclave, the Eminent Commander mentioned something in his discussion about the five libations in the Order of the Temple that stuck out in my mind, and I thought I'd share my thoughts here today.  As we are all very aware, the first three libations are to Solomon King of Israel, Hiram King of Tyre, and Hiram Abiff; all serving to connect us to the Symbolic or Blue Lodge.
	
	The fourth libation however, is the one that stuck out in my mind.  As you recall, that libation is ``to the memory of Simon of Cyrene, who was compelled to bear our savior's cross.'' and is intended to unite us with the Earthly teachings and Crucifixion of Christ for the remission of sin.  This struck me as compelling because of all of the people and events recorded throughout the New Testament, including the apostles, why was Simon of Cyrene of such significance to be both mentioned in three of the four Gospels, and to be heralded in our sacred tradition.
	
	One thing that bible study has shown me is that any mention, whether it is lengthy and repeated or simple and quick, it has significance to be pondered, studied, and understood.
	
	To that end what else do we know of Simon of Cyrene?  Mathew 27:32 only tells us of his name and origin.  Luke 23:26 speaks only to his being ``on his way in from the country.''  We learn the most in Mark 15:21 in which it is told that he was the father of two sons named to us as Alexander and Rufus in a manner which leads one to believe that these names were known in some way at the time of its writing some 70 years after the crucifixion.  While it has been disputed, many offer that Simon was a Jew who likely came to Jerusalem for Passover and brought his sons with him.  We also see reference in Romans 16:13 of a greeting to ``Rufus, chosen in the Lord'' who is suggested to be that same son of Simon.
	
	Based on what I have found through research and conversations with a friend who is a Methodist Pastor and religious scholar, I believe that Simon of Cyrene was the first instance, and a symbol for us all, of the beginning of a transition of responsibility for Christ's teachings from Jesus to mortal man.  This may have been the start of our responsibility to the world.  It was amid Jesus' suffering at the hands of worldly evil that Simon was compelled to come forward and bear the burden of carrying the cross.  Perhaps this was a handoff from Master to Apprentice.  Jesus knowing that his final hours were at hand, knowing that he would no longer be able walk among man and teach God's true word, needed help from man, beginning that moment and extending into time eternal to continue sharing those lessons with the world.  In Luke 14:27 we are told by Jesus that ``whosoever doth not bear his cross, and come after me, cannot be my disciple.''  This was the calling which was begun and symbolically illustrated when Simon was called to help Jesus.
	
	There can be no doubt that this was a life altering experience for Simon individually.  We do not have any written record as such, but in my mind I cannot fathom that such an experience in which Simon and Jesus had to work together to carry the cross up the hill, in which Simon bore witness to Jesus' willingness to suffer for the sake of man, did not change Simon, or his sons forever.  I believe that this was a moment that also changed the world and man's obligation to God.  We at that moment became not just students and listeners, but teachers ourselves.  We became the stewards now committed to carry forth Jesus's teachings throughout the world.
	
	In Acts 2:10 we learn that the people of Cyrene were among those converted at Pentecost and later dispersed due to persecution yet were the foundation of Christian teaching at Antioch.  While there is no record to document it, I am compelled to believe that Simon and his sons would have returned to Cyrene, located in modern day Libya, and spread word of their experience and perhaps of their own belief in our blessed savior which may have spread and was later to be confirmed through Pentecost. 
	
	Sir Knights, my intent today is not to present this as a formal report of fact or even an historical study.  Rather I wanted to share my thoughts with you, my brothers, to provoke thought, and share my own, as we journey together as students and Disciples of Christ hoping that we might learn together and become stronger in our faith through our brotherhood together.
	
	\begin{quote}
		``We all have a cross to bear; let each of us so bear that cross that we may be deemed worthy to wear the crown.''
	\end{quote}
	
\end{document}