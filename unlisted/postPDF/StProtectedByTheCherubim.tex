\documentclass[letterpaper,11pt]{article}

\usepackage{geometry}
\usepackage{xcolor}
\usepackage{fancyhdr}
\usepackage{graphicx}
\usepackage{parskip}
\usepackage{amsmath,amssymb}
\usepackage{textcomp}

\title{Protected by the Cherubim}
\author{Companion Gabriel Jagush, PTIM\\Texas Council \textnumero{} 321, RSM}
\date{\textit{Originally presented at the November 6, 2019\\Stated Assembly of Texas Council \textnumero{} 321, RSM}}

\begin{document}
	
	\maketitle
	
	1 Kings 6:14,19:
	
	\begin{quote}
		So Solomon built the house, and finished it. And the oracle he prepared in the house within, to set there the ark of the covenant of the \textsc{Lord}.
	\end{quote}

	When Solomon finished the \textit{Sanctum Sanctorum} with the help of Hiram Abiff, he seated the Ark of the Covenant beneath the extended wings of the Cherubim. When he completed and dedicated the Temple to the glory of God as we witnessed in the Most Excellent Master's degree, the glory and name of God manifested itself as a cloud of flame and black smoke. This fulfilled the promise that God made to King David.
	
	What do the Cherubim and the fiery presence of God on their wings mean for us as masons, though?
	
	\section*{The Cherubim in Scripture}
	
	The cherubim fulfilled several functions: they protected the ark of the covenant, they touched the inner walls of the temple, and they supported the presence of God. In 1 Chronicles 28:18, they are referred to as God's chariot. This was known as the ``Mercy Seat,'' and in Greek was called the \textit{hilasterion}, or ``that which removes sin.''
	
	Chapters 1 \& 10 of the book of \textit{Ezekiel} explain that the Cherubim are used by God as his chariot to move around Chaldea and to leave the temple in Jerusalem. The concept of God's chariot was and is considered so important in Judaism that there is an entire school of thought known as \textit{Merkabah Mysticism} or \textit{Chariot Mysticism}.
	
	The central theme of Chariot Mysticism is stories of ascent to heaven, and God's presence on His throne. This theme manifests in the school's primary practice, which is called the \textit{Work of the Chariot} and focuses on meditation and contemplation. In the \textit{Work of the Chariot}, practitioners go through a series of veils, each guarded by an angel, and secured by a password and sign. The password and sign for each veil was the name and signet of the angel guarding it.
	
	As the practitioner passes through each veil, he unfolds more and more of God's divine revelation, and re-joins his soul with God. This process, including passing the veils using passwords and signs, should be very familiar to us as Royal Arch Masons.
	
	The process of reintegration with God is also a central theme of Jewish and Christian mysticism. Revelation 22:14 states:
	
	\begin{quote}
		Blessed are they that do his commandments, that they may have right to the tree of life, and may enter in through the gates into the city.
	\end{quote}

	This ties back to Genesis 3:24:
	
	\begin{quote}
		So he drove out the man; and he placed at the east of the garden of Eden Cherubims, and a flaming sword which turned every way, to keep the way of the tree of life.
	\end{quote}

	Once again, we see the Cherubim as guardians --- this time, as guardians of the Garden of Eden. The Tree of Life is the key to both the Garden of Eden and the city of New Jerusalem (assuming they are not, in fact, the same place). In order to enter New Jerusalem, or reintegrate with God, we must follow its path.
	
	The Tree of Life details ten different stages that man must reach in order to reintegrate with God, followed in a specific order (as detailed by the flaming sword of Genesis). Traveling through ten stages means that there are nine veils, gates, or archways to pass, much like the nine arches that we learn about in the Royal Arch Mason and Select Master degrees.
	
	\section*{The Cherubim in Ritual}
	
	In the Royal Master degree, the Cherubim are involved from the start. The candidate is received by walking into the room, under the extended wings of the Cherubim, and around the Ark of the Covenant. This is different from his reception in the Capitular degrees in that it he is not formally received, and his conductor does not give him an explanation of how and why he is being received in the way that he is. It is also more of a symbolic act on the candidate's part than an instruction from the conductor.
	
	The cherubim that extend their wings over the Ark of the Covenant are a representation of God's strength and protection, because they both support God's presence, and protect the Ark of the Covenant. For a candidate, every degree is a request for further light. When the candidate passes under the wings of the Cherubim, and into the circle of angels, he is showing that any quest for more truth must first start by placing oneself in the strength and protection of God.
	
	We know that while they supported the name and fiery presence of God with one set of wings, the Cherubim also touched the inner walls of the Temple with their other set of wings, but could not be seen from outside the Sanctum Sanctorum. This is very important to us as Freemasons, because King Solomon's temple represents us as human beings. As the strength, protection, and chariot of God, the Cherubim connect the inner walls of our mental, emotional, and spiritual temple to the presence of God that is inside each and every one of us, even though nobody can see that from outside of our hearts. Our inner spiritual lives are guarded from the world.
	
	The Tree of Life can be divided into three pillars that are named Severity, Mercy, and Harmony. The two Cherubim on the Ark of the Covenant can be seen as representing Severity and Mercy, while the fiery presence of God represents the pillar of Harmony. The top of the pillar of Harmony in the Tree of Life is known as the Crown, and represents total integration with God. The Cherubim and Ark from Scripture and our ritual are a reminder that the Tree of Life --- which is our key to Eden and New Jerusalem --- is inside of us.
	
	\section*{In Conclusion}
	
	As Royal Arch Masons and Royal \& Select Masters, we are performing our own \textit{Work of the Chariot}. Every degree we have taken, and every degree we confer, is one more veil that we leap through. The more we learn about ourselves, and the more we improve ourselves through Freemasonry, the closer and closer we come to reintegrating with the Divine fire that lives both inside of us and all around us.
	
	The Cherubim remind us that God lives in every one of us. Our hearts are filled with his fire and presence. We know the name of God --- we just have to look within ourselves and surrender to His strength and protection.
	
\end{document}