\documentclass[letterpaper,11pt]{article}

\usepackage{geometry}
\usepackage{xcolor}
\usepackage{fancyhdr}
\usepackage{graphicx}
\usepackage{parskip}
\usepackage{amsmath,amssymb}
\usepackage{textcomp}

\title{Preserving the Word}
\author{Companion Gabriel Jagush, PTIM\\Texas Council \textnumero{} 321, RSM}
\date{\textit{Originally presented at the October 2, 2019\\Stated Assembly of Texas Council \textnumero{} 321, RSM}}

\begin{document}
	
	\maketitle
	
	\section*{The Common Work}
	
	The American \textit{Royal Master} and \textit{Select Master} degrees of the Cryptic Council originally have their origins as offshoots of \textit{Ecossais} degrees from the Burgundy region of France. These degrees all deal with the method of preserving the “true Word.” They are known as “Cryptic” degrees, “Ninth Arch” degrees, “Secret Vault” degrees, or “Sacred Vault” degrees. They all share common themes: there is always a treasure, and there is always a vault, cave, or underground hiding spot.
	
	In the original legend, the Builder of the Temple engraved the Word on a triangular neck jewel that he always kept on his person. In a moment when his life was endangered, he threw the jewel - and consequently, the Word - into a well in the northeast corner of the Temple. Much time later, three workers or builders found the jewel in the well at noontime. Later, the legend was changed so that the hiding of the Word was pre-planned and deliberate.
	
	This legend appears in the Jewish Talmud and the Christian Bible, and is found - in some form - in every Masonic Rite all over the world. One of the first references to the Cryptic legend is found in \textit{Anderson's Constitutions of 1723}, which details the preservation of secrets as revealed to Enoch - something we learn about in the Royal Arch degree. Enoch's story in the Sacred History bridges the gap between the Capitular degrees, which deal with recovering the Word, and the Cryptic degrees which deal with preserving the degrees. Our Cryptic degrees themselves take place a long time after Enoch's story.
	
	We know there were two vaults - Enoch's and Solomon's. Solomon's builders found Enoch's vault, and Solomon later constructed his own separate vault where he deposited the Word. Early versions of the Select Master degree treat of both arches, but our present-day version only deals with Solomon's arches.
	
	Throughout the world in every Rite, the Craft degrees only deal with the loss of the Word, and the Capitular degrees only deal with the restoration of the Word.
	
	All versions of the Select Master degree have common themes:
	
	\begin{enumerate}
		\item There is always a treasure.
		\item There is always a vault, cave, or underground hiding place.
		\item There is always loss of the knowledge of God, either by the fall of Adam, the wickedness of mankind after the flood, or the plot against Hiram.
		\item There is always a method of preservation, either by hollow pillars, an ark, or a messiah figure.
	\end{enumerate}
	
	The manner of preservation doesn't really matter --- what matters is that they are all places of refuge or safe places to hide secrets. The ark itself is an innovation from later versions of the degree, and is entirely made-up for the purpose of our degrees.
	
	\section*{The Treasure}
	
	The historical object of the Cryptic degrees is to detail how a secret treasure was hidden - in our case, the idea was developed by Hiram Abiff, and executed by King Solomon. In a literal sense, the secret treasure is the physical manifestation of the Word, or the name of God, on top of the Ark of the Covenant. However, as Masons, we should understand that what is being preserved is the \textit{knowledge} of the Ineffable Name of God, not necessarily a representation or written form thereof. In a narrative sense, it is as important to modern Masons as ever, because without knowing how the word was preserved, the Royal Arch degree doesn't make sense. It is the completion of the story, and the way we fill in all the “plot holes” in the degrees. More importantly, however, it is a proof and lesson that as Masons, we should desire to know God above all else, and treasure His name.
	
	\section*{The Secret Vault}
	
	Descending into the secret vault can represent many things. When I think of the more light-hearted parallels, I think of how passing each arch is like a passing year in, or deeper level of, a friendship, love of a spouse, trust of a brother, or knowledge gained. However, that's not the primary purpose of the secret vault.
	
	The secret vault is, emblematically, a grave, and by descending into it, we emulate the process of dying. Zabud encounters the point of no return when he goes through the door that was left ajar. After that, he must either die, or join the Select of the Twenty-Seven, but he can never un-open that door. Just as Zabud went past the point of no return, so will we, eventually. We can't undo becoming Select Masters, and we certainly can't reverse death.
	
	However, what awaits us in the secret vault is the Word. Here on earth, we can only guess at what God is really like, or what awaits us after we die. Much as Zabud is totally clueless until he descends into the vault and becomes one of the twenty-seven, we don't know what awaits us until we embrace death and join the ranks of those that have died. How different would this story be if Zabud had rejected Solomon's offer and been executed? How different will our story be if we reject God's truth?
	
	\section*{The Discovery}
	
	Discovering the sacred treasure in the Royal Arch degree and discovering the secret vault in the Select Master degree represents an allegory for the effect that discovering the nature of God has on our lives. The Word itself was used for the government of the Craft, and seen as being so important that it needed to be preserved. After preservation, it survived the destruction of the Temple and was hidden away safely until it was discovered by the three Most Excellent Masters of the Royal Arch degree, who found it by being willing to do any work, even if they risked their lives by doing so. The Word is so important that the Select of the Twenty-Seven were willing to defend it at all costs, and were more deeply connected to each other than anyone else in the Craft. The Word completes our Masonic journey.
	
	In the same way, knowledge of God governs our lives, and it's so important to us as a species that we've done everything we can to preserve it, be it through telling others, or by writing holy texts. God, and knowledge of God, survives every calamity and every disaster. God is always waiting for us to find Him, and we should pursue Him at all costs - even if we risk our lives. This pursuit binds us together as believers more closely than anyone else. Much how the Word completes our Masonic journey, God completes our lives, and much how the Word was hidden in the Temple, knowledge of God is hidden inside each and every one of us.
	
\end{document}