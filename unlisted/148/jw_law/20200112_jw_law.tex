\documentclass{article}

\usepackage{geometry}
\usepackage{amsmath,amssymb}
\usepackage{textcomp}
\usepackage{parskip}

\newcommand{\gmd}[2]{
	Grand Master's Decision \textnumero{} #1, #2
}

\title{Minute by Minute}
\author{Gabriel Jagush\\Junior Warden}
\date{January 12, 2020\\Fort Worth Lodge \textnumero{} 148\\Stated Meeting}

\begin{document}
	\maketitle
	
	There are three primary laws regarding the minutes of a lodge, all under Title II, Chapter 19 of the Laws of the Grand Lodge of Texas:
	
	\begin{itemize}
		\item Article 337, ``Minutes of Lodge''
		\item Article 338, ``Inspection of Minutes: By Whom''
		\item Article 338a, ``Summary of Minutes Permitted'' 
	\end{itemize}
	
	The summary of Article 337, ``Minutes of Lodge,'' is as follows:
	
	\begin{itemize}
		\item The permanent record of the minutes has to be in a bound book or a locking ring binder.
		\item You have to write your minutes in a format as close as possible to that provided by Grand Lodge.
		\item Your minutes have to be accurate and complete (which is kind of a no-brainer).
		\item The minutes must be signed by the Worshipful Master and Secretary in person at the meeting in which they are approved.
		\item The minutes must list the officers who were there and the number of members and visitors present, whose names go in the register.
	\end{itemize}

	Here's a summary of Article 338, ``Inspection of Minutes: By Whom:''
	
	\begin{itemize}
		\item Lodge minutes are private records.
		\item You can't make any copies of them for anything that's not masonic.
		\item The secretary has to keep them secure and preserved.
		\item They need to be available for inspection at ``reasonable times'' by officers, members, and appropriate visitors, but never non-masons.
	\end{itemize}

	Article 338a, ``Summary of Minutes Permitted,'' is actually the most detailed out of all three of these articles:
	
	\begin{itemize}
		\item You're allowed to prepare and publish a summary of the minutes and records of the lodge as long as it doesn't discredit the Fraternity, the lodge, or any of its members past or present.
		\item The summary must be presented to and approved by the lodge members prior to publication.
		\item You can use a wide variety of sources to compile this summary, both from the lodge, and the Grand Lodge.
		\item You have to use the term ``Masonic Disciplinary Violation'' instead of ``Masonic Offense.''
		\item You can't include the names of any petitioners, candidates, Entered 	Apprentice Masons, or Fellowcraft Masons.
		\item You can't publish the names of rejected candidates.
		\item 
		\item You can't publish any libel.
		\item The law provides a suggested outline for the summary.
	\end{itemize}

	Additionally, There are two other articles in Chapter 19 that deal with the minutes:
	
	\begin{itemize}
		\item Article 280 specifies that the Worshipful Master is responsible for the correctness of the minutes.
		\item Article 333 states that the first order of lodge business should be to read, correct, and approve the minutes.
	\end{itemize}
	
	There are a number of Grand Master's decisions that deal with the minutes:
	
	\begin{itemize}
		\item \gmd{8}{1978} --- The IRS has a right to look at your lodge minutes.
		\item \gmd{5}{2001} --- The minutes of the lodge must be permanently stored at the lodge.
		\item \gmd{16}{2004} --- It is not permissible to give a copy of lodge minutes to the county library, even if the lodge is demised.	
		\item \gmd{3}{2006} --- A traveling lodge, or ``trunk lodge,'' may only meet if it and its paraphernalia, including its minutes, are determined to be ``masonically secure'' by a District Deputy Grand Master.
		\item \gmd{9}{2011} --- Because the charter and minutes must be kept secure at all times, member of the lodge shall be present in the lodge at all times that non-masons have access to the lodge property.
	\end{itemize}
	
	
\end{document}