\documentclass[letterpaper]{article}

\usepackage{geometry}
\usepackage{parskip}
\usepackage{url}


\title{Form 990-N (e-Postcard)}
\author{}
\date{\vspace{-4em}}


\begin{document}
	\maketitle
	
	Filing requirements have been changed by the Internal Revenue Service.  All subordinate Chapters of the Grand Chapter of Texas that do not have income of \$50,000 must file the Internal Revenue Service Form 990-N (e-Postcard) on the internet each tax year beginning with the  tax year ending on or after December 31, 2007.  The IRS has ruled that since the fiscal year of the subordinate Chapters is June 24 to June 23, for tax purposes the end of the fiscal year for subordinate Chapters is \textbf{June 30}, since they do not recognize partial months.
	
	The 990-N (e-Postcard) must be filed every year by the 15th day of the 5th month after close of the tax year.  Therefore, Chapters this year must file their 990-N no later than \textbf{November 15}, each year.  You cannot file the e-postcard until after your tax year ends; therefore, DO NOT file the form until AFTER June 30, 2012.  THE 990-N CAN ONLY BE FILED ELECTRONICALLY.  NO PAPER COPIES ARE SENT TO THE IRS!
	
	To file the 990-N (e-Postcard), go to the IRS Web Site at \url{www.irs.gov} and find the Annual Electronic Filing Requirement for Small Exempt Organizations by entering in the search box – Form 990-N (e-Postcard).  Click on the designated spot and you will be directed to the IRS trusted partner, Urban Institute, to file the 990-N.  You will register to file as a new user and you will create a login ID and password.  Be sure to retain these in a safe place to use next year.
	
	To complete the 990-N, you will need the following information:
	
	\begin{enumerate}
		\item The Chapter Employer identification number (EIN).  Each Chapter has a unique EIN and it is on the letter you received from the IRS.  If you have any questions about your number, call the Grand Secretary’s office to get the number assigned to your Chapter by the IRS.  It will be a nine-digit number ``00-0000000.''
		\item Tax year (July 1, 2011 to June 30, 2012).
		\item Legal name and address.  You must use ``Royal Arch Masons of Texas Grand Chapter (No.) (Chapter Name)''.  Chapter Mailing Address.
		\item Any other names the Chapter uses.
		\item Name and address of a principal officer, secretary or treasurer.
		\item Web Site address if the Chapter has one.
		\item Confirmation that the Chapter’s annual gross receipts are less than \$50,000.
	\end{enumerate}
	
	Please note that a Chapter that fails to file a 990-N (or 990 for Chapters with income of over \$50,000) for three consecutive years will automatically lose its tax-exempt status. \textbf{ Each Chapter has a tax id number so call if you need yours. (254-753-6721)}
	
	\textbf{Please provide the Grand Secretary’s office a copy of the report that your Chapter filed with the IRS, Form 990-N (or 990), on or before November 15th at \url{gractx@aol.com} or mail P O Box 296, Waco, TX 76703.} \pagebreak
	
	\section*{Useful Information}
	
	All Chapters and Councils must file some type of Form 990 each year (beginning in 2007)
	
	It is the responsibility of the Chapter or Council to go to the IRS website and file the return.
	
	The gross receipt test includes all income from every source, such as dues, degree fees, rents, royalties, interest, dividends, donations, building fund contributions, etc. Those Chapters and Councils with less than \$50,000 in gross income must file the 990-N electronically. This is done at \url{www.irs.gov/990n}
	
	To begin, a ``user account'' has to be created. Print out each section as you go (Ctrl +P) so that you can keep the individual questions and information for filing each year.
	
	Once the user account is created, the EIN for the Chapter and Council can be added and filed by the same user account. (So if you file the 990n for the lodge, chapter, council and commandery, etc., they can all be filed under the same user account)
	
	The fiscal year ends June 23, which means you would use the reporting year of July 1 to June 30 for this report, with the return being due on November 15th.
	
	All of the Subordinate Chapters are titled: Royal Arch Masons of Texas, with a dba of XXXXXXXXX Chapter No. YYY
	
	All of the Subordinate Councils are titled: Royal and Select Masters of Texas, with a dba of XXXXXXXXX Council No. YYY
	
	Forward a copy of the e-mail received stating that the 990-n has been received and accepted by the IRS to the Grand Secretary-Recorder.
	
	3. If gross receipts are more than \$50,000 in a year, either a Form 990 or 990EZ will be required from your Chapter and /or Council, although there is no actual Income Tax liability, as they are exempt from tax under Section 501 ( c ) ( 8 ) of the Internal Revenue Code. The latest exemption ruling date is March 13, 1975, the Activity Code is 260.
	
	The following information is furnished to aid in the preparing of Form 990.
	
	\begin{enumerate}
		\item Date of Exemption Letter: March 13, 1975
		\item Group Exemption Number: Chapter \#0182, Council \#1339
		\item Exemption Code Paragraph: 501 ( c ) ( 8 )
		\item Activity Code: 260
	\end{enumerate}

	If you cannot file the 990-N, and due to the complexities in the tax form and regulations it is recommended that the Chapter and/or Council contact a Certified Public Accountant or other tax professional concerning the completion of the required forms.
	
\end{document}