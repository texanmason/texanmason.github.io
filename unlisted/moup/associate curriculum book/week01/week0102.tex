\chapter{Martinism in Ten Points}

\subsection*{What do you mean by ``esoteric?'' What is the ``work'' within \moup{}?}

The adjective ``esoteric'' applies to that which is hidden, not obvious, as opposed to
``exoteric'' meaning outward. Esoteric teachings go beyond the mere rational or
philosophical education. The teacher is to imbue the soul of the student with a desire to
become a disciple; thereby enabling the student to exceed his own personality so that the
spirit can manifest. Esotericism leads mankind to the doors of the Knowledge of Self. In
this sense, the Traditional Sciences, such as Alchemy, Astrology, Kabbalah or the Tarot, are powerful aids but neither necessary nor complete.

\begin{quotebox}
And this is the point where so many Seekers get lost. Often a Seeker possesses what has
been called by some the desire, but lacks access to a proper initiator. The result is a Brother lost in the labyrinth, and in this state of confusion and darkness the Brother delves deeply
into occultism, too deeply into mere occultism, never arriving at the threshold, or worst
still, mistaking the traps of his own design, taking false-things as the real-threshold;
enlarging their ego with the belief that they are indeed powerful, authoritative, and in some
cases, divine, and ultimately becoming as dead in the world.
\end{quotebox}

So, if the glamorous and sexy practices of fundamentally exoteric practices and schools of
mere occultism are incomplete and unnecessary, what are more correct practices for one a
genuine desire or calling to this threshold of knowledge?

On this question, all human experience is in absolute agreement on the answer: disciplined
prayer (not the begging and pleading often confused as prayer), purification of the
multifaceted-self, detachment, and discernment, and the outward giving to others.

Thusly, we who want to achieve Knowledge seek it through prayer, purification,
detachment and the practice of discernment, and in raising and helping others. \moup{} offers to help any sincere seeker in performing 
selfless work on oneself. The practice of charity, prayer, kindness and compassion towards
others, study and deepening of Esoteric Knowledge of the Hidden Science and Tradition
gradually brings the neophyte on the Path to apprehend his life more subtly.

\subsection*{What type of work does \moup{} offer?}

The teachings of \moup{} focus on individual work which touches three areas:
\begin{itemize}
    \item The mind, through the intelligent observation of nature and its phenomena, the
study of classical occultism (what we call today esotericism) of what is around you
and yourself, as well as the development and exercise of a finer discernment. The
content of the Martinist teaching is based on the fundamental concepts of Christian
esotericism, the notion of ritual and occult constitution of man.
    \item The feeling, by awakening to the Universal Tradition, Christians principles and
messages of Passed Masters. This spiritual and mystical alchemy that occurs
gradually in the sincere seeker of the soul will be manifested in several forms: a
desire to help and love your neighbor first in his inner circle before they finally
realize that humanity is a single thing as well as an inner transformation in order to
find the Primordial Truth to give life its real meaning to get out of egocentrism to
expand its field of consciousness.
    \item Action by the practice of this awareness that invites prayer and meditation. This
ongoing practice will lead the practitioner to more subtlety and sensitivity, without
falling into sentimentality that would slow him before the onset of emotional
outbursts often unproductive. The action is then reflected in the practice of selfless
service to the world around us, seeking the help of Heaven for man finally realizes
that he can do nothing without God's help.
\end{itemize}

\subsection*{Who are the Passed Masters to which reference is made?}

These Passed Masters are many and date back to Pythagoras, through Plato, Plutarch,
Meister Eckhart, Jacob Boehme, Swedenborg, Eckhartshausen, Martines Pasqually and his
disciple Louis Claude de Saint Martin. Each of these characters has marked his time and
made his stumbling block to Tradition. Maître Philippe de Lyon is the Guide selected by the
Martinist Order of Unknown Philosophers which is placed under his holy protection.
His disciples, including John and Chapas Michel Saint-Martin, and all those who inspired
them or which were later inspired by them are some references to which the Order is
attached.

If like other Martinist Orders, the Order of Unknown Philosophers considers \textbf{PAPUS} (Dr
Gérard Encausse) as the founder of the Martinist Order, it looks upon Master \textbf{PHILIPPE
LYON} as Head of Martinism, who said, ``When Papus leaves, it will be
a bit of a mess in Martinism. There will be several groups, but I will take Martinism and I
will be the Grand Master.''

\subsection*{What kind of initiatory approach do you suggest?}

Initiatory approach proposed by \moup{} is very simple
and is based primarily on the will and desire of the neophyte to undertake its own
transmutation. Besides the study, meditation, prayer, and work on oneself, to make his
character more and more transparent and responsive to inner values, it is left to the
researcher entire freedom to think. It is in his daily life, during his own experiences, he can
check and test the veracity of ancient texts and Sacred Words, and the benefits that it will
be drawn from these teachings will be left to his own judgment. Such values do not make
noise and do not sparkle. They are not always recognized, certainly. But this process,
simple, humble, and intelligent leads to practice charity, tolerance and kindness to all
beings, and a sense of inner peace and compassion will accompany all actions.

\subsection*{What conditions are required to study the esoteric?}

The Spirituality should be considered a ``plus'' in the context of family, social and
professional balance. In no case can not be a substitute for shortages or any loophole. No
condition is required for its study. A lively intelligence, intellectual curiosity, open-mindedness associated with a pure heart free from fanaticism and egocentrism bring all
naturally seeking to question life and purpose. In no event family life must not be
considered as a source of imbalance or disharmony because Martinism must help to BE in
life, not flee.

\subsection*{How should one address the areas of the unseen?}

The first thing, when it comes to the areas of the invisible, is to keep well grounded, the
benevolent spirit, humble and sincere attitude and heart open to others. Do not get carried
away by false promises offering you a radical transformation after a period of ten days. Part
of the Martinist Order Unknown Philosophers helps you to undertake individual work
seeking common sense, intelligent introspection and constant attention to the why of our
actions, so that what we are, what we say and what we do are always in harmony.

\subsection*{What are the education and training courses you offer?}

Our teaching is as much about the Hermetic tradition and, what is usually meant by Occult
Science, as the work on yourself, so that everyone accesses, to the extent possible, self-knowledge and that of others. ``Know thyself and you will know the universe and the gods,''
already said Socrates. Putting this knowledge into practice, and trying first to understand
and meet our own shortcomings and failures, we will thus contribute so that love prevails
among them, and thus contribute so that love reigns among men: Peace to Men of good will!

\subsection*{What kind of schools do we develop to better help others?}

We have many candidates who want to help others in the field or as soon as they believe
they have developed certain faculties. But do they really have these faculties --- and in this
case the discernment --- to help others without impinging on their freedom? This is not
easy! Do they risk, contrary to their good faith, some damage, given their lack of
experience?

We believe that we must first work on ourselves to improve ourselves by studying, meditating and praying. Thus, little by little, armed with new knowledge, and enlightened by our Past Masters, we will be able to try to help our neighbor in the little things and affairs of everyday life, first and foremost, while respecting the plans of Heaven. because only He knows what is good for all of us. And it is by transforming oneself, placing oneself more in \textit{being} than in \textit{having}, that we will be able to act far from useless long speeches.