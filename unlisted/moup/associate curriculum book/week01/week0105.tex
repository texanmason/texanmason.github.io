\chapter{The Marrakech Convention, 1988}
\section{Introduction}
\textit{This is an excerpt of the orginal document.}

This Manual elaborates and clarifies exactly why one should be a Martinist, what Martinism is, the purpose of a Martinist Lodge and the proper procedures to be followed by Martinist Lodges chartered in the name of the SUPR\hexdot{} CONS\hexdot{} INT\hexdot{} DE L’OR\hexdot{}. This Manual is in two sections: \textbf{Section I – GENERAL}, for lodge Officers and all members of the Order; \textbf{SECTION 2 – SPECIFIC}, for Lodge Officers only. 

PREAMBLE. After Pasquales passed on, the temples of the Elus-Cohen eventually
closed. L. Claude de Saint-Martin, an Elu Cohen and member of the Order of Unknown
Philosophers, continued to transmit the \si{} initiation to persons he deemed worthy.
They in turn initiated others. This activity was not carried on under the aegis of an
organized body but was perpetuated on a person to person (or \textit{free}) basis. Papus received The
Initiation from Henri Delaage and later found out that several of his friends had also
received it via different channels. Papus felt that The Initiation was much too precious for
its perpetuation to be left to individual transmissions; therefore, he founded \textit{the Martinist
Order} principally to secure and ensure its survival. During the years leading up to the
formation of the first Supreme Council in 1980, Papus founded in 1888 the magazine
\textit{L’Initiation} as the organ through which teachings and information about the Order would be
disseminated.

The advent of World War II brought about the cessation of almost all fraternal
activity in Europe. The members were scattered as a result of the turmoil. Indeed, Grand
Master Chevillon was assassinated by the Nazis. Several of the traditional esoteric Orders
also met their demise. Fortunately, the Martinist Order survived. The O.M.\&S. was the
only branch of the Order that has operated continuously, being active during the war years
in neutral Switzerland. After the war, activity gradually restarted in France and in French-speaking countries but progress in English-speaking countries was slow, due largely to the
lack of information in English. In the early 1980’s, the Order began operations in the
Western Hemisphere under the auspices of the Britannic Grand Lodge, and a serious effort
at rebuilding in English-speaking countries began. The International College of Martinist
Studies was inaugurated and charged with distributing information and teaching in English,
working in parallel with \textit{L’Initiation}, which continues its work in French. The influx of
information now available in English has stimulated and rekindled a great interest in Martinist Work in
English-speaking countries, worldwide. This Manual of Information and Instruction has therefore been
provided so that standard and uniform procedures can be set and maintained in all jurisdictions. 

\begin{paperbox}{For More Information}
    Martinist teachings and those of the Esoteric Arcana are available from:
    
    The International College of Martinist Studies\\
    P.B. Box W31, Worthington, Barbados
    
    Write for a copy of the current book list.
\end{paperbox}