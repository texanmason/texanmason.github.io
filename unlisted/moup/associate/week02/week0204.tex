\section{Second Conventicle}

In our first conventicle as Associates we were introduced to the study of symbolism. Brethren are
expected to dwell upon the subject of our work between sessions so as to become more familiar with
their Martinism.

Symbols are the working tool of the mystic. Proficiency in their use leads to attunement with the
Hierarchy whence these mystical symbols originate. Our Order encourages members to use all
opportunities for mentally reviewing and meditating upon the symbolic stages of the Path of Initiation,
whether traveling to daily work, sitting quietly in a park or elsewhere amongst plant life-especially
trees, or in a home Oratory.

We now come to our heritage of Mystical Symbols. These are rarely the creation of one person, except
perhaps the Master who found an Initiatory Circle or Esoteric Order. Their purpose is to enshrine an
aspect of Eternal Truth. First we shall examine the emblems conferred upon us at our Initiation, which
are of the strict Martinist tradition. Then we are to consider the signs, symbols, and emblems used in
conventicles of the Associate Degree. Lastly, we complete this preliminary and important fundamental
study with an analysis of the Martinist Pantacle used on official documents of the Order, such as
charters and certificates.

The remainder of the Associate Degree is concerned with the mystical symbolism which is shared by
three other great currents of Western Initiation, namely the Masonic, the Pythagorean, and the
Rosicrucian. Common ground exists between the Occidental Orders of Initiation, so the accomplished
Martinist may occasionally visit other esoteric Temples and receive honorary initiation upon
recommendation of the Grand Master of this Jurisdiction. 

{\centering
	(Short discussion if desired) 
}

One of the first Martinist emblems placed upon the Neophyte is the black mask. The initiator speaks
these significant words:

\begin{quote}
	By this symbol your personality disappears: you become an Unknown in the midst of other
	Unknowns; you have no more to fear the little susceptibilities to which daily life is constantly
	subject among beings always interested in finding you at fault; you are well guarded against
	the snares that \textit{ignorance} joined to \textit{conceited opinion} will lay every day against you. On the
	contrary, as our ancient Brethren, apply yourself in secrecy while observing the others. 
	
	Let the Mask of circumspection ever protect you against the inquisitive looks of those
	whose character and behaviour have not proved them worthy to come and appear in the
	Sacred Sanctuary where Truth delivers her oracles.
	
	Finding yourself alone before people that you do not know, thou you have no favour to ask
	of them: it is from yourself in all your loneliness that you must grasp the principle of your
	own advancement. Expect nothing from others except in case of absolute need; in other
	words, learn to be yourself.
	
	Unknown, you have no orders to receive from any one. You alone are responsible for your
	acts before yourself; and your Conscience is the Master to be feared, from whom you must
	always receive counsel - the Judge inflexible and severe, to whom you must render a just
	account of your acts.
	
	This Mask, which isolates you from the rest of mankind during the period of work, shows
	you the price that you must attach to your Liberty, almighty by the Will before Destiny and
	before Providence. 
	
	``That liberty,'' as said Eliphas Levi, ``which one may call the Divinity of Man, the most
	beautiful, the most superb, the most irrevocable of all the gifts of God to man. That liberty
	which the Supreme Creator himself could not violate without denying His own nature; that
	liberty which one ought to obtain by force when he does not possess it as a supreme
	autocracy.''
	
	And, O my Brother/Sister, you do not possess that liberty, which is the liberty of the soul
	and mind, and not merely that of the body; it is by fighting against your passions, your earthly
	cravings, that you may hope to conquer that freedom so praised, so exalted, so truly Divine.
	
	No one upon Earth is capable of depriving you of that intellectual and moral liberty; you
	alone art absolute Master of it, and you alone will answer before your Creator and your God
	for the errors and faults that ignorance may have caused you to commit. 
	
	 Let the Mask teach you to remain unknown to those you have saved from misery or saved
	from ignorance. Know how to sacrifice your worldly personality whenever the welfare of the
	collectivity may command it. 
	
	Buddha, the great teacher of morals and ethics, teaches in most sublime strains the doctrine
	of Nirvana, or self-denial and self-effacement. This doctrine of extreme self-abnegation
	means nothing more than the subjugation and conquest of our carnal self. For you know that
	Man is a composite being. In him he has the angelic and the animal, and the spiritual training
	of our life means no more than the subjugation of the animal and the setting free of the
	angelic.
	
	These are, in other terms, the teachings of so profound a symbol as the Mask; still other
	applications will be revealed if thy heart truly desires them. 
	
	This symbol is the foundation-stone of Martinism, and we represent it hieroglyphically by
	the letter \hebyod{} (``yod''), because this letter is the principle, the cellule, from which all the letters of
	the Hebrew alphabet are formed; and the masked Associate is also the principle, the cellule,
	which forms the great body of the temporal and spiritual regenerated Humanity. 
	
	The Mask is also represented by the figure 10, it being the number of the letter \hebyod{} (``yod'') and
	the number of Thought, both human and Divine. 

\end{quote}

From this explanation we learn of the first great step along the Path. The Quest of Man is often
repeated in the old Delphic adage: ``Know thyself and thou shalt know the Universe and the Gods.''
Our Founder Master, Martinez Pasquales writes of the ``Re-integration of Beings.'' The fulfillment of
human evolution lies in expanding the limited everyday consciousness to reach all planes or
emanations of Omneity. This is a work of many lifetimes. We have all Eternity before us, but there is
not a moment to lose. We are not securely on the Path until an inward dedication takes place.
Whence? How? Whither? Man? 

Note the emphasis on ``collectivity.'' The Initiate Paul, who was raised by the Master when on his way
to Damascus (Acts 9:3-9), subsequently taught: ``Ye are members, one of another.'' Our Pythagorean
brethren say ``My Brother is my other self.'' So the deep symbolism of the mask is to moderate the
mundane personality so as to encourage an inner reliance. Far from inhibiting the personality, the
symbol points to a redirection of the personal powers for the collective good. How often do we hear or
read of Service to Mankind only to find that the ``Do-gooders'' are primarily concerned with attracting 
attention to themselves or their organizations. The lesson of the mask is one of Silent Service, which
allows the Martinist to listen for inner guidance. Rushing around, beating drums as it were, is as
wasteful of opportunity as not troubling to find one’s Path in life. The former is misdirection. We must
Orient ourselves on the Middle Way of balanced attention, listening to the Heart and using the Head. 

Our venerated Master, Louis Claude de Saint-Martin wrote: 

\begin{quote}
	I have desired to do good but I have not desire to make noise, because I have felt that noise did no
	good, and that good made no noise.
\end{quote}

(Instructor now says: ``Let us meditate and hold this thought. I will repeat it slowly\ldots'')

\subsection{Exercise A}

Find a quiet space in your home. If you can manage an Oratory, or a place which can be
dedicated to meditation, so much the better. It should be furnished simply – a chair, a small
table, and a candle. You may burn some incense if you like. Make sure there are no pictures of
living things (human, animal or vegetable) to distract you.

Now you can begin to discipline your mind for the Great Journey out of the Forest of Errors, as
Saint-Martin described the general manner of living and thinking.

Put on your alb and sash, and don your mask. Darken the room, and seat yourself about four
feet from a mirror. Place a single lighted candle between yourself and the mirror. Relax for a
few minutes, then ask yourself the question put to you at your Initiation:

``We do not ask who you are, because if you knew, you would have nothing more to learn; but
we do ask you whom do you think you are?''

Do not discuss this exercise until you have practiced it several times.
