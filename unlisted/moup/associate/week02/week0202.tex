\section{Exercise B}

Sit comfortably in a chair or lie down on a couch. Relax your body completely, close your eyes and follow, for five minutes, the course of your thoughts, which you will try to remember. At first you will notice that the types of thoughts that will rush upon you involve questions of daily life, your occupation, worries and so forth.

With respect to these thoughts, assume the role of a quiet observer, free and independent.

According to your state of mind at the time and the situation of the moment, you will either find this exercise easy or very difficult. In both cases it is important not to lose the course of your thoughts, nor to forget yourself, and to follow attentively.

You must be careful not to fall asleep during this exercise. If you feel tired, it is preferable to stop at once and postpone the exercise, therewith making a resolution not to be tired next time.

This control of thought must be practiced in the mornings and evenings. Every day you must extend its duration by half or one minutes, so that after two weeks you are able to observe the course of your own thoughts for ten minutes without the slightest deviation. The most important thing is to be conscientious and precise, since there is no need to rush these exercises. This development is individual and therefore different for each person. However, it is of no use to go to the next step until this one if fully mastered