\documentclass{article}

\usepackage{geometry}
\usepackage{parskip}

\begin{document}

\section*{Excercise A}

Find a quiet space in your home. If you can manage an Oratory, or a place which can be
dedicated to meditation, so much the better. It should be furnished simply --– a chair, a small table, and a candle. You may burn some incense if you like. Make sure there are no pictures of living things (human, animal or vegetable) to distract you.

Now you can begin to discipline your mind for the Great Journey out of the \textit{Forest of Errors}, as Saint-Martin described the general manner of living and thinking.

Put on your alb and cordelier, and don your mask. Darken the room, and seat yourself about four feet from a mirror. Place a single lighted candle between yourself and the mirror. Relax for a few minutes, then ask yourself the question put to you at your Initiation:

``We do not ask who you are, because if you knew, you would have nothing more to learn; but we do ask you whom do you think you are?''

Do not discuss this exercise until you have practiced it several times. 

\section*{The Martinist Visualization \& Prayer}

It should be done \textit{daily}: on awakening; on getting ready for sleep at night --- and many
times during the day.

Be comfortable. Close your eyes if circumstances allow, relax and breathe easily. Visualize yourself in your alb; see it radiantly white; call to mind its meaning --- the Original Robe of Glory --- purity.

Put on your cordelier, symbol of the Magic Circle and the Traditional Chain. It reminds you of your linkage to your Initiator, to all others in the Chain of Light, and to the Source of Light Itself. 

Now enfold yourself in your mystical black cloak. Tt renders you insensitive to the attacks of the base powers of ignorance. Through its use, you enter easily into meditation; you ``go within'' and commune with your innermost self, which knows all. 

Lastly, don your mask, the most powerful tool for the development of the true mystical personality. Through its use you subjugate the petty ego and protect yourself from the undesirable aspects of your mundane personality such as pride, conceit, vanity, and arrogance.

Visualize above yourself the flaming Pentagram. 

Mentally review yourself now, clearly see yourself completely dressed in your
habiliments. Actually \textit{feel} them about you. See the flaming pentagram above you and \textit{hear} the flames hissing; \textit{feel} their warmth. 

Now, silently or very softly, intone the Name \textit{Ieschouah} (yea-hesh-shoe-wah) one or more times.

After a short period, using your feeling nature, strive to be \textit{en rapport} with the Past Masters and all brethren in the Initiatic Chain. Call to mind someone near and dear to you such as a family member, a friend, or someone you wish to help: a world leader, a nation or group of people. Visualize them fully clothed in the vestments. Visualize the Pentagram flaming above them. Again intone the Name \textit{Ieschouah} one or more times (either silently or softly) and send them thoughts of love.

Remember: the Higher Powers care little for the shallow values of the profane world, but everything for love, brotherhood, kindliness charity, humility, sincerity, beneficence, consolation, and compassion.

Never neglect this visualization exercise. It is a prayer of the strongest potency. Let it be a routine, several times daily --- an essential practice of your Martinist Way of Life. Use this Mystic practice as you move through the day, silently blessing all those whom you observe to be saddened, distraught, sick, in need. Your meditations will indicate to you how to use this visualization and prayer in other ways. 

\section*{Exercise B}

Sit comfortably in a chair or lie down on a couch. Relax your body completely, close your eyes
and follow, for five minutes, the course of your thoughts, which you will try to remember. At
first you will notice that the types of thoughts that will rush upon you involve questions of daily
life, your occupation, worries and so forth.

With respect to these thoughts, assume the role of a quiet observer, free and independent.

According to your state of mind at the time and the situation of the moment, you will either find
this exercise easy or very difficult. In both cases it is important not to lose the course of your
thoughts, nor to forget yourself, and to follow attentively.

You must be careful not to fall asleep during this exercise. If you feel tired, it is preferable to
stop at once and postpone the exercise, therewith making a resolution not to be tired next time.

This control of thought must be practiced in the mornings and evenings. Every day you must
extend its duration by half or one minutes, so that after two weeks you are able to observe the
course of your own thoughts for ten minutes without the slightest deviation. The most important
thing is to be conscientious and precise, since there is no need to rush these exercises. This
development is individual and therefore different for each person. However, it is of no use to go
to the next step until this one if fully mastered. 

\section*{Prayer for the Dedication of the Home Oratory or Altar}
{\large\textbf{Morel, S::I::G::I::}}

We humbly beg you, Almighty and Eternal God, by Your Only Son, Jesus Christ, Our Master and
Lord, to sanctify \textit{[make cross motion using right hand]} with Your celestial blessing this Altar (Oratory) destined for pure and holy uses.
As in the past You received the prayers and respects of Israel, wandering in the Desert, in a
Tabernacle to which You gave Your Servant, Moses, the service and care, I beg you to consider
this humble Altar (or Oratory) arranged for Your Glory and Your Service. May you, O Lord of the
Sky and the Earth, instill the same virtue that You gave in the past to your Saint of Saints, and
may Your celestial blessing spread in this instant and this place on him, so that Your servants who
will meet around this sacred surface be then sanctified by the Celestial Virtue of the divine
mysteries which will be celebrated there, and may they give them protection of their bodies and
souls, and view the eternal life. Through Jesus Christ, our Lord and Master, and through Saint
John, His servant. Amen. \textit{[make cross motion using right hand while saying, ``Amen.'']}

\section*{The Heart of Love}
{\large\textbf{by Servitor S::I::I::}}

The Heart of Love is a method of meditation and prayer that works directly with the most
profound and powerful teachings of Jesus: 

\begin{quote}
	You shall love the Lord your God with all your heart, with all your soul, and with
	all your mind. This is the first and great commandment, and the second is like it,
	that you shall love your neighbor as yourself.
\end{quote}
\begin{quote}
	As you have done to the least of these my brethren, you have done to me.
\end{quote}
\begin{quote}
	This is my commandment, that you love one another as I have loved you.
\end{quote}

Quite naturally, this method resonates strongly with many of the clearest statements of
Louis Claude de Saint-Martin:

\begin{quote}
For our personal advancement in virtue and truth one quality is sufficient,
namely, love; to advance our fellows there must be two, love and intelligence; to
accomplish the work of humanity there must be three love, intelligence, and
activity. But love is ever the base and the fount in chief.
\end{quote}
\begin{quote}
Hope is faith beginning; faith is hope fulfilled; love is the living and visible
operation of hope and faith.
\end{quote}

It takes very little reflection to know this method involves everything taught in the quotes
above: Love for God and our fellow souls is at once the motive force, the immediate
sentiment, and the practical aim of this work. It makes use of our human intelligence in
the most graceful and subtle of ways --– an internal cultivation of love that stimulates more
of the ``living and visible operation of hope and faith'' in our relationships with others. It
puts the Martinist Way of the Heart into immediate practice.

This method is developed through four phases. It is recommended that the first phase be
practiced alone for at least a week, and then each phase can be successively added over a
period of several weeks until you are finally practicing all four phases in each sitting.
Once a working familiarity has been developed with each phase, then the practitioner
may place more or less emphasis on various phases, and even rearrange them, as desired.
Some people might find this method suitable as the mainstay of their regular devotionals
and inner work, while others might prefer to use it less routinely. This method is an
excellent practice for anyone who wishes to serve in spiritual healing, for it helps in
keeping one’s soul open to the flow of higher energies and tends to infuse one’s healing
prayers with the special sweetness of selfless love.

\subsection*{Phase One: The Heart of Love Received from the Exemplar}

It is always advisable to begin and end such work with a ritual action such as lighting a
candle and perhaps some incense, and performing the Qabalistic Cross or ordinary
crossing. After settling into a centered and peaceful state of meditation, offer a prayer of
submission to the Divine Will, expressing your desire to know and serve it through love.

Next, call to mind the image of someone you consider to be a great historical
embodiment and exemplar of love, such as Jesus, Mother Mary, St. Francis of Assisi,
Mother Theresa, or Rachel the Jewish matriarch. Imagine this person standing in front of
you with a loving smile. See within his or her chest a flaming heart, radiating love out
through the whole body in rich hues of pink, ruby and golden light, like a splendid
sunrise. 

Feel the warmth on your face and chest. Let yourself respond emotionally to this great
soul’s love, smiling in return. Imagine your exemplar reaching out to cup your heart in
his or her hands, and the flames of love flowing into and igniting your own heart. If you
feel moved to weep with gratitude, allow that to happen as you continue to meditate upon
this person as an embodiment of Divine Love, a living vessel through which God loves
the world, including you. To accept this love is itself an act of love for God, for the
exemplar, and for yourself. You may speak with your exemplar if you wish.

In your meditation, consider that to ancient people the heart was not merely symbolic of
emotions, but was also the seat of intuition, inspiration, beauty, peace and harmony. There is much to discover here about the nature of love, which includes far more than our
feelings of affection and sympathy. Consider these words of Saint-Martin:

\begin{quote}
	The head of old was subject to the ruling of the heart, and served only to enlarge
	it. Today the scepter that belongs of right to the heart of a person has been
	transferred to the head, which reigns in place of the heart. Love is more than
	knowledge, which is only the lamp of love, and the lamp is less than that which it
	enlightens.
\end{quote}

When you are ready to end the meditation, simply let the image fade. Offer a final prayer
of thanks and return your consciousness to the external world, though now infused with
an elevated awareness of love.

While most people report this exercise to be positive and uplifting, some people may also
find themselves challenged by various kinds of discomfort with the work. For example,
feelings of unworthiness, guilt or shame may arise. It is important to simply be aware of
all our feelings, both pleasing and uncomfortable, accepting them as indicators of deeper
processes occurring within our hearts and minds. In effect, they present us with
opportunities to learn more of what we really believe about ourselves and our
relationships with the Divine. In response to such observations, it is important to
remember that accepting the infinite grace of Divine Love is not about using the head to
strategize a path toward righteous worthiness, but is rather about simply opening the heart
to the immediate fact of God’s freely given mercy and affection. With this
understanding, where we find self-condemning thoughts and feelings of self-loathing, we
have the opportunity to practice acceptance, forgiveness and healing of our own
humanity, as well as truly nurturing ourselves toward more virtuous living.

\subsection*{Phase Two: The Heart of Love Shared with Those We Cherish}

Proceed through the previous phase and just past the point where your heart is ignited by
the exemplar. Allow the image of the exemplar to fade, and in its place imagine someone
among your friends and family with whom you share a deep bond of love. Perhaps this is
someone you know to be in extra need of receiving love at this time. See him or her
smiling in the warmth of the pink, ruby and golden light radiating out through your body.
Imagine yourself reaching forward to hold that person’s heart in your hands. See and feel
the flames of your heart flowing through your arms to ignite his or her heart with love.
Speak with this person if you wish. Meditate upon the love you have shared, how it has
been expressed between you, and how it might grow.

When you are ready, allow that person’s image to fade. If you feel moved to do so, allow
the image of another cherished friend or family member to arise, and then repeat the
entire process. You can continue through as many loved ones as you wish, eventually
ending the meditation as before.

As with the previous phase, this can be a very touching and joyful exercise, and yet it can
also prove challenging. In focusing on your love for another, you might discover areas of uncertainty or sense something lacking. For example, you might realize that in some way
you have not been as expressive of your love and affection as you might be. This could
be due to various fears or inhibitions for either or both of you. You might also discover
you have resentments, frustrations or other negative feelings about the individual that
seem to prevent you from more fully and freely loving him or her. As you practice the
exercise with different people in mind, you may become more aware of how your love
differs from one person to another. With some people your sentiments might be more
affectionate, with others more appreciative or admiring, while for others more
compassionate or sympathetic. In any case, this phase of the Heart of Love can help you
learn about how you feel, think and behave in your relationships with loved ones, and
thus provide you with many opportunities to refine your ability to love each person in
your life in a way as unique and meaningful as he or she is.

\subsection*{Phase Three: The Heart of Love Shared with Those Who Challenge Us}

Work through the first two phases, and now begin extending your love toward someone
you feel has mistreated or offended you in some way, or someone you have difficulty
trusting. Give just as freely and energetically to this soul as you did in the second phase.
Meditate upon the many pearls of wisdom in loving those we may not find easy to love.
Ponder how you might manifest love for this person more outwardly. As before, repeat
the process until you are ready to end the meditation.

\subsection*{Phase Four: The Heart of Love in All}

After working through all the previous phases, meditate upon the universe as existing
within the Flaming Heart of God, the One in whom we live and move and have our
being. Recall that your heart is aflame with that same Divine Fire, and that it is actually a
spark of that Divine Fire, as are all the hearts of God’s children. Allow all the
implications of meaning, virtue and action to flow freely through your heart and mind,
with neither resistance nor attachment, but with awareness, acceptance and love.

\end{document}