\section{The Heart of Love}

\textit{by Servitor S::I::I::}

The Heart of Love is a method of meditation and prayer that works directly with the most
profound and powerful teachings of Jesus: 

\begin{quote}
	You shall love the Lord your God with all your heart, with all your soul, and with
	all your mind. This is the first and great commandment, and the second is like it,
	that you shall love your neighbor as yourself.
	
	As you have done to the least of these my brethren, you have done to me.
	
	This is my commandment, that you love one another as I have loved you.
\end{quote}

Quite naturally, this method resonates strongly with many of the clearest statements of
Louis Claude de Saint-Martin:

\begin{quote}
	For our personal advancement in virtue and truth one quality is sufficient,
	namely, love; to advance our fellows there must be two, love and intelligence; to
	accomplish the work of humanity there must be three love, intelligence, and
	activity. But love is ever the base and the fount in chief.
	
	Hope is faith beginning; faith is hope fulfilled; love is the living and visible
	operation of hope and faith.
\end{quote}

It takes very little reflection to know this method involves everything taught in the quotes
above: Love for God and our fellow souls is at once the motive force, the immediate
sentiment, and the practical aim of this work. It makes use of our human intelligence in
the most graceful and subtle of ways – an internal cultivation of love that stimulates more
of the “living and visible operation of hope and faith” in our relationships with others. It
puts the Martinist \textit{Way of the Heart} into immediate practice.

This method is developed through four phases. It is recommended that the first phase be
practiced alone for at least a week, and then each phase can be successively added over a
period of several weeks until you are finally practicing all four phases in each sitting.
Once a working familiarity has been developed with each phase, then the practitioner
may place more or less emphasis on various phases, and even rearrange them, as desired.
Some people might find this method suitable as the mainstay of their regular devotionals
and inner work, while others might prefer to use it less routinely. This method is an
excellent practice for anyone who wishes to serve in spiritual healing, for it helps in
keeping one’s soul open to the flow of higher energies and tends to infuse one’s healing
prayers with the special sweetness of selfless love.

\subsection*{Phase One: The Heart of Love Received from the Exemplar}

It is always advisable to begin and end such work with a ritual action such as lighting a
candle and perhaps some incense, and performing the Qabalistic Cross or ordinary
crossing. After settling into a centered and peaceful state of meditation, offer a prayer of
submission to the Divine Will, expressing your desire to know and serve it through love.

Next, call to mind the image of someone you consider to be a great historical
embodiment and exemplar of love, such as Jesus, Mother Mary, St. Francis of Assisi,
Mother Theresa, or Rachel the Jewish matriarch. Imagine this person standing in front of
you with a loving smile. See within his or her chest a flaming heart, radiating love out
through the whole body in rich hues of pink, ruby and golden light, like a splendid
sunrise.

Feel the warmth on your face and chest. Let yourself respond emotionally to this great
soul’s love, smiling in return. Imagine your exemplar reaching out to cup your heart in
his or her hands, and the flames of love flowing into and igniting your own heart. If you
feel moved to weep with gratitude, allow that to happen as you continue to meditate upon
this person as an embodiment of Divine Love, a living vessel through which God loves
the world, including you. To accept this love is itself an act of love for God, for the
exemplar, and for yourself. You may speak with your exemplar if you wish.

In your meditation, consider that to ancient people the heart was not merely symbolic of
emotions, but was also the seat of intuition, inspiration, beauty, peace and harmony. 

There is much to discover here about the nature of love, which includes far more than our
feelings of affection and sympathy. Consider these words of Saint-Martin:

\begin{quote}
	The head of old was subject to the ruling of the heart, and served only to enlarge
	it. Today the scepter that belongs of right to the heart of a person has been
	transferred to the head, which reigns in place of the heart. Love is more than
	knowledge, which is only the lamp of love, and the lamp is less than that which it
	enlightens.
\end{quote}

When you are ready to end the meditation, simply let the image fade. Offer a final prayer
of thanks and return your consciousness to the external world, though now infused with
an elevated awareness of love.

While most people report this exercise to be positive and uplifting, some people may also
find themselves challenged by various kinds of discomfort with the work. For example,
feelings of unworthiness, guilt or shame may arise. It is important to simply be aware of
all our feelings, both pleasing and uncomfortable, accepting them as indicators of deeper
processes occurring within our hearts and minds. In effect, they present us with
opportunities to learn more of what we really believe about ourselves and our
relationships with the Divine. In response to such observations, it is important to
remember that accepting the infinite grace of Divine Love is not about using the head to
strategize a path toward righteous worthiness, but is rather about simply opening the heart
to the immediate fact of God’s freely given mercy and affection. With this
understanding, where we find self-condemning thoughts and feelings of self-loathing, we
have the opportunity to practice acceptance, forgiveness and healing of our own
humanity, as well as truly nurturing ourselves toward more virtuous living. 


\subsection{Phase Two: The Heart of Love Shared with Those We Cherish}

Proceed through the previous phase and just past the point where your heart is ignited by
the exemplar. Allow the image of the exemplar to fade, and in its place imagine someone
among your friends and family with whom you share a deep bond of love. Perhaps this is
someone you know to be in extra need of receiving love at this time. See him or her
smiling in the warmth of the pink, ruby and golden light radiating out through your body.
Imagine yourself reaching forward to hold that person’s heart in your hands. See and feel
the flames of your heart flowing through your arms to ignite his or her heart with love.
Speak with this person if you wish. Meditate upon the love you have shared, how it has
been expressed between you, and how it might grow.

When you are ready, allow that person’s image to fade. If you feel moved to do so, allow
the image of another cherished friend or family member to arise, and then repeat the
entire process. You can continue through as many loved ones as you wish, eventually
ending the meditation as before.

As with the previous phase, this can be a very touching and joyful exercise, and yet it can
also prove challenging. In focusing on your love for another, you might discover areas of uncertainty or sense something lacking. For example, you might realize that in some way
you have not been as expressive of your love and affection as you might be. This could
be due to various fears or inhibitions for either or both of you. You might also discover
you have resentments, frustrations or other negative feelings about the individual that
seem to prevent you from more fully and freely loving him or her. As you practice the
exercise with different people in mind, you may become more aware of how your love
differs from one person to another. With some people your sentiments might be more
affectionate, with others more appreciative or admiring, while for others more
compassionate or sympathetic. In any case, this phase of the Heart of Love can help you
learn about how you feel, think and behave in your relationships with loved ones, and
thus provide you with many opportunities to refine your ability to love each person in
your life in a way as unique and meaningful as he or she is.

\subsection{Phase Three: The Heart of Love Shared with Those Who Challenge Us}

Work through the first two phases, and now begin extending your love toward someone
you feel has mistreated or offended you in some way, or someone you have difficulty
trusting. Give just as freely and energetically to this soul as you did in the second phase.
Meditate upon the many pearls of wisdom in loving those we may not find easy to love.
Ponder how you might manifest love for this person more outwardly. As before, repeat
the process until you are ready to end the meditation.

\subsection{Phase Four: The Heart of Love in All}

After working through all the previous phases, meditate upon the universe as existing
within the Flaming Heart of God, the One in whom we live and move and have our
being. Recall that your heart is aflame with that same Divine Fire, and that it is actually a
spark of that Divine Fire, as are all the hearts of God’s children. Allow all the
implications of meaning, virtue and action to flow freely through your heart and mind,
with neither resistance nor attachment, but with awareness, acceptance and love.