\section{Third Conventicle}

Respected Associates:

Three other items of investiture during your Associate initiation consisted of the slippers, the white alb
or robe/surplice, and the sash with the Martinist Pantacle. 

The slippers you had upon yourself upon first entering the Temple Space; their white color symbolize
the purity which should guide our footsteps on the initiatic path, and indeed the casting off and
replacing of yet another piece of temporal clothing and replacing it with a blessed or sacred garment or
vestment reminds us of the injunction given to Moses in Exodus 3:5, where Moses is told to remove his
temporal footwear: ``Do not come any closer,'' God said. ``Take off your sandals, for the place where
you are standing is holy ground.'' We have removed ours and now bear blessed and consecrated
covering as a part of our \textit{vestments} as symbolically we enter the sacred and consecrated Temple Space
erected by His Holy Angels who, prior to your entry, we entreated to build a spiritual Temple for us,
through which His Force and Blessing may spread on His people.

The Alb (albus in Latin means ``white''), represents the highest gradation of light and has been worn by
mystics of all ages to denote such meaning; in fact, it has given rise to the term ``\textit{Great White
Brotherhood'}'. Devoting our attention to this alb or robe, at first we learn of the outer meaning; that
all who wear it enter the temple as equals, without any distinction as to social distinction or sex. We
enter the Temple humbly seeking a new life. Secondly, in the Temple we approach the Light as pure
beings, as children of the Light. We have temporarily left our Karma at the portal, which we will later
resume and perfect in the outer world. Finally it represents our aura which become strengthened and
purified by living the Initiatic Life. It is by our labors that we gradually replace this symbolic vestment
with the real vestment of Light – the ROBE OF GLORY. As is common in initiatic societies, neophytes
have often been called a \textit{candidate}. This term, derived from the Latin word, ``\textit{candidatus},''
(pronounced: Kan-dih-DAH-toos), which means, ``to be clothed in white'', for in many of the ancient
ceremonies the initiates were clothed with a white robe following their lustration. This robe or alb,
with its white color symbolizing these things, along with the slippers and sash and Pantacle, also
represents the first elements of spiritual armor with which you gird yourself on beginning the initiatic
path (Ephesians 6:10-18) within our Order: 


\begin{quote}
	Finally, be strong in the Lord and in his mighty power.$^{11}$ Put on the full armor of God, so that you can
	take your stand against the devil’s schemes. $^{12}$ For our struggle is not against flesh and blood, but
	against the rulers, against the authorities, against the powers of this dark world and against the
	spiritual forces of evil in the heavenly realms. $^{13}$ Therefore put on the full armor of God, so that when
	the day of evil comes, you may be able to stand your ground, and after you have done everything, to
	stand. $^{14}$ Stand firm then, with the belt of truth buckled around your waist, with the breastplate of righteousness in place, $^{15}$ and with your feet fitted with the readiness that comes from the gospel of
	peace. $^{16}$ In addition to all this, take up the shield of faith, with which you can extinguish all the flaming
	arrows of the evil one. $^{17}$ Take the helmet of salvation and the sword of the Spirit, which is the word of
	God. 
\end{quote}

You are also protected with the white sash, its color denoting your rank amongst your B.: and SS.:, as
hierarchy and order is also a lesson being taught within this grade; a lecture of hierarchy was
mentioned to you in the initiation ritual, and has been given to you. It is a difficult thing for many to
accept that some must follow in life, and some must lead; that to prepare ourselves to lead others we
must be willing to take upon ourselves the yoke of an unyielding conscience, constantly re assessing
ourselves to make sure that all the proper and painful steps are taken to bear this responsibility for
others; and to prepare ourselves to follow, we must be receptive to the lessons taught, not by the
dogma of another human being but by the Divine Spark or Spirit which guides us all, even when, or
*especially* when our carnal instincts tell us otherwise. Lytton illustrates this point in his work Zanoni,
when the disciple asks his master if the cruel disparities of life will ever be done away with. The
answer is given thus: 


\begin{quote}
	''Disparities of the PHYSICAL life? Oh, let us hope so. But disparities of the INTELLECTUAL and
	the MORAL, never! Universal equality of intelligence, of mind, of genius, of virtue!--no teacher left to
	the world! no men wiser, better than others,-- were it not an impossible condition, WHAT A
	HOPELESS PROSPECT FOR HUMANITY! No, while the world lasts, the sun will gild the mountaintop
	before it shines upon the plain. Diffuse all the knowledge the earth contains equally over all
	mankind to-day, and some men will be wiser than the rest to-morrow. And THIS is not a harsh, but a
	loving law,--the REAL law of improvement; the wiser the few in one generation, the wiser will be the
	multitude the next!'' 
\end{quote}

For this reason, the rank denoted by its color is also complemented by it being draped from the left
shoulder to the right hip, denoting that we are giving of ourselves in the form of our ego, personal
wants and identity, which along with the mask teach us to self-abnegate for the sake of the collectivity
(as opposed to the left side of receptivity). This giving of ourselves or this self-sacrifice will be
exemplified in a fashion in the following ``Exercise B''. For this reason also, the great symbol of our
Order, the Martinist Pantacle, is placed in a talismanic position on this side as a form of symbolic
protection (though not in the superstitious sense). This Pantacle will be elaborated on in later degrees. 

\subsection{EXERCISE B}

Sit comfortably in a chair or lie down on a couch. Relax your body completely, close your eyes
and follow, for five minutes, the course of your thoughts, which you will try to remember. At
first you will notice that the types of thoughts that will rush upon you involve questions of daily
life, your occupation, worries and so forth.

With respect to these thoughts, assume the role of a quiet observer, free and independent.
According to your state of mind at the time and the situation of the moment, you will either find
this exercise easy or very difficult. In both cases it is important not to lose the course of your
thoughts, nor to forget yourself, and to follow attentively.

You must be careful not to fall asleep during this exercise. If you feel tired, it is preferable to 
stop at once and postpone the exercise, therewith making a resolution not to be tired next time.
This control of thought must be practiced in the mornings and evenings. Every day you must
extend its duration by half or one minutes, so that after two weeks you are able to observe the
course of your own thoughts for ten minutes without the slightest deviation. The most important
thing is to be conscientious and precise, since there is no need to rush these exercises. This
development is individual and therefore different for each person. However, it is of no use to go
to the next step until this one if fully mastered.********************************************** 