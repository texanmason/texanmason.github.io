\chapter{First Conventicle}

Those who seek admission to \mouplong{} are known as ``Neophytes'' and as such are brought by a responsible brother or sister to face an interrogation by a representative tribunal of the Order. When asked questions based upon the traditional requirements of effective membership we may well wonder what the purpose of this interrogation really is. The First Degree of our venerable Order is designed to clear the ground for building or perhaps rebuilding our symbolic temple. So, before the Chapter Master and his advisory officers accept new members they must look for signs that the Neophyte displays a dedicated purpose. Likewise, it may be as well for all to review those questions before seeking the test of the Second Degree, which completes the Pronaos of Martinism. 

Exoteric Orders with the Keys of Initiation make full use of \textit{symbols}. Now, a symbol is a sign by which
something is known. It may take many forms. For example; we find as an exoteric symbol the color
red for danger. In ancient writings a circle symbolized the Sun. An emblem, or garment worn, which
conveys a special meaning to either the beholder or the wearer, is a symbol. All language, both spoken
and written, is symbolized by sounds or visual signs. In the final analysis our very thinking takes place
in mental symbols formed out of the impressions passed to the brain through our five physical senses. 

In Martinism we classify symbols under three headings, namely: 

\begin{itemize}
    \item Mystical symbols: for example, a cross
    \item Artificial symbols: for example, an alphabet
    \item Natural symbols: for example, smoke
\end{itemize}

Our Order aims at encouraging members to think for themselves, then to discipline their thought so as
to stimulate an expansion of consciousness, consequently lesson material is kept short, leaving time for
both discussion and meditation at each Conventicle. Any idea that a long series of discourses or
readings is conducive to spiritual advancement should be abandoned. Experience has taught many
seekers that intellectual studies can easily become excessive and so fog the Path. Useful reading as a
helpful background will be recommended from time to time but pure Martinism rests upon the study
of Two Symbolic Books in the next two Degrees. Let us therefore concentrate now on the learning and
practicing our \textit{alphabet of symbolism} so that our \textit{ultimate reading} may be thorough and worthwhile.
For discussion and meditation, until we meet again, we are to consider example of the three kinds of
symbols. 

